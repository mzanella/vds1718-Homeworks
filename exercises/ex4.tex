\newcommand{\exFour}
{$\wbS{b}{S}~\cong{}_{sos}~\wbS{b}{(\ifABC{b}{S}{\skipistr})} $}

\newcommand{\exFourIff}
{$\confSs{\wbS{b}{S}}{s}\Rar{*}s'~\iff{}~\confSs{\wbS{b}
{(\ifABC{b}{S}{\skipistr})}}{s}\Rar{*}s'$}

\newcommand{\exFourLtR}
{$\confSs{\wbS{b}{S}}{s}\Rar{*}s'~\Rar{}~\confSs{\wbS{b}
{(\ifABC{b}{S}{\skipistr})}}{s}\Rar{*}s'$}

\exercise{Esercizio 4}
{
	Prove or disprove the following semantic equivalence:\\
	\begin{center}
	\exFour{}
	\end{center}
	where $b\in\bexp$ and $S\in\stm$
}
{
	Per dimostrare 
	\begin{center}
	\exFour{}
	\end{center}
	posso dimostrare che 
	\begin{center}
	\exFourIff{}
	\end{center}
	dimostrando separatamente i due versi dell'implicazione. \\
	
	\textbf{Implicazione \Rar{}} \\
	Per dimostrare che 
	\begin{center}
	\exFourIff{}
	\end{center}
	posso utilizzare l'induzione sul numero di passi che la derivazione di 
	\confSs{\wbS{b}{S}} impiega per arrivare ad una configurazione finale 
	(supponendo, per ipotesi, che a partire dallo stato $s$ \wbS{b}{S} termini).
	\\
	\casobase{n=0} Questo caso è banalmente vero poichè 
	\confSs{\wbS{b}{S}}{s} non è una configurazine finale ed in 0 passi quindi
	non può evolvere in una configurazione finale. Lo stesso ragionamento si
	può applicare nel caso in cui \textbf{n=1} e \textbf{n=2} poichè,
	sfruttando l'unfolding, sono neccessari almeno 3 passi per arrivare ad una
	configurazione finale.
	\casobase{n=3} In questo caso 
}
