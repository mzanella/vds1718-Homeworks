\newcommand{\lvar}[1]{\textrm{\normalfont{\texttt{lvar(#1)}}}}
\newcommand{\exTwo}{$if~\confSs{S}{s}\Rar{*}s'~then~\forall{} x \notin lvar\{S\}.s(x)=s'(x)$}

\exercise{Esercizio 2}
{
\begin{enumerate}
\item Define formally a function $lvar:\stm\rightarrow\wp(\var)$ such that
for any $S\in\stm$, \lvar{S} is the set of variables in \var{} that appear
in the left-hand side of some assignment that occurs in the statement S.
\item Prove that for any $S\in\stm, s,s'\in\states:$ \\
\begin{align*}
if \confSs{S}{s}\Rar{*}s' then \forall{} x \notin lvar\{S\}.s(x)=s'(x)
\end{align*}
\end{enumerate}
}
{
\textbf{Parte 1}.
Per definire la funzione \texttt{lvar} ne do una definizione composizionale. Gli elementi base sono:
\begin{itemize}
\item \lvar{\skipistr} = $\emptyset$
\item \lvar{\memupdate{x}{a}} = \{x\}
\end{itemize}
Gli elementi composti invece:
\begin{itemize}
\item \lvar{\concat{$S_1$}{$S_2$}} = \lvar{$S_1$} $\cup$ \lvar{$S_2$}
\item \lvar{\ifABC{b}{$S_1$}{$S_2$}} = \lvar{$S_1$} $\cup$ \lvar{$S_2$}
\item \lvar{\wbS{b}{S}} = \lvar{$S$}
\end{itemize}

\textbf{Parte 2}. Per dimostrare \exTwo{} utilizzo l'induzione sulla struttura
di $S\in\stm$
I casi base sono:
\begin{itemize}
\item \casobase{S=\skipistr}: Se S=\skipistr{} allora comunque prenda $s\in\stm$
ho che $\confSs{\skipistr}{s}\Rar{}s$. Per definizione di \skipistr{} quindi lo stato s non viene modificato e quindi qualsiasi variabile non verrà alterata.

\item \casobase{S=\memupdate{x}{a}}: Se S=\memupdate{x}{a} allora per
definizione dell'update in memoria si ha che 
$\confSs{\memupdate{x}{a}}{s}\Rar{}s[x\rightarrow\mathbb{A}[a]_s]=s'$ e quindi
per definizione dell'update in memoria si ha che 
$\forall{}y\in\var~.~y\notin\lvar{\memupdate{x}{a}}$ y non verrà aggiornata e
quindi $s(y)=s'(y)$
\end{itemize}
}

\newpage