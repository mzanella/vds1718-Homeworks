\newcommand{\whilem}{\syntacticCategory{While\textsuperscript{-}}}
\newcommand{\exOne}{$\forall{}S\in\whilem,\forall{}s\in\states.\exists s'\in\states.\confSs{S}{s}\Rar{*}s'$}

\exercise{Esercizio 1}
{
	Let us consider the language \whilem{} defined by removing from \while{}
	the iterative command \texttt{while}.  \\
	Prove that: 
	\begin{center}
	for any 
	$S\in\whilem,$
	$ \forall{}s\in\states.\exists s'\in\states.\confSs{S}{s}\Rar{*}s'. $
	\end{center}
}

{
	Per provare che
	\begin{center}
		\exOne{} 
	\end{center} 
	utilizzo l'induzione sulla struttura di $S\in\stm$.
	
\paragraph{Casi Base} 
	\begin{itemize}
	
	\item \casobase{S=\skipistr}: 
	\casespace{}
	
	Se S=\skipistr{} allora per dimostrare lo
	statement è sufficiente applicare la regola dello \skipSOS{} e quindi preso
	un qualunque $s\in\states$ si ha:
	\begin{align*}
	\confSs{\skipistr}{s}\Rar{}s
	\end{align*}
	E quindi per determinismo si ha che $s=s'$ e quindi S termina in un passo.
	\postcasespace{}

	\item \casobase{S=\memupdate{x}{a}}: 
	
	\casespace{}
	
	Se S=\memupdate{x}{a} con $x\in\var$
	e $a\in\num$ allora per dimostrare lo statement è sufficiente effettuare
	l'update della memoria con la regola \assSOS{} e si ha che lo statement
	termina.
	\\
	Quindi preso un
	qualunque $s\in\states$ si ha:
	\begin{align*}
	\confSs{\memupdate{x}{a}}{s}\Rar{}s[x\rightarrow{}\mathbb{A}[a]_s]=s'
	\end{align*}
	E quindi S termina in un passo.
	\postcasespace
	\end{itemize}

\paragraph{Casi Induttivi} 	
I casi induttivi sono la \textbf{concatenazione} e l'\textbf{\texttt{if}}. 
	\begin{itemize}

	\item \casoinduttivo{S=\concat{$S_1$}{$S_2$}}:	
	\casespace{}
	\\
	
	Se S=\concat{$S_1$}{$S_2$}	con $S_1,S_2\in\whilem$, allora per ipotesi
	induttiva so che: 
	\[ \forall{}s\in\states, \confSs{S_1}{s}\Rar{*}s' \qquad  \textrm{e} 
	\qquad \confSs{S_2}{s}\Rar{*}s''\quad \textrm{qualunque sia s} \]
	Poichè 
	\[ \confSs{S_1}{s}\Rar{*}s' \quad \textrm{(per ipotesi induttiva)} \]
	allora 
	\[ \exists{}k\in\mathbb{N}  \quad \textrm{tale che}  \quad \confSs{S_1}{s}
	\Rar{k}s'\]
	e quindi per il \textbf{lemma di composizione}
	\[ 	\confSs{\concat{$S_1$}{$S_2$}}{s}\Rar{k}\confSs{S_2}{s'} \] 
	Poichè	dato un qualunque stato so che per ipotesi induttiva che  
	$ \confSs{S_2}{s'} $ 
	termina allora, in particolare, $\confSs{S_2}{s'}\Rar{*}s''$.

	\item \casoinduttivo{S=\ifABC{b}{S_1}{S_2}}: 	
	\casespace{}
	\\
	
	Se S\textbf{=\ifABC{b}{$S_1$}{$S_2$},} allora per ipotesi induttiva so che: 
	\[ \forall{}s\in\states, \confSs{S_1}{s}\Rar{*}s' \quad \textrm{e} \quad 
	\confSs{S_2}{s}\Rar{*}s'' \]. Quindi posso dimostrare che 
	\confSs{\ifABC{b}{S_1}{S_2}}{s}\Rar{*}$s'$ per casi:
	\begin{itemize}
	\item \fbox{$\mathbb{B}[b]_s=tt$ } Allora per la regola \ifttSOS{} abbiamo:
	\[ \confSs{\ifABC{b}{S_1}{S_2}}{s}\Rar{}\confSs{{S_1}}{s} \] e poichè
	sappiamo per \hi{} che \[ \confSs{S_1}{s}\Rar{*}s' \quad 
	\textrm{ qualunque sia s} \] si ha che 
	$\confSs{\ifABC{b}{S_1}{S_2}}{s}\Rar{*}s'$
	\postcasespace
	\item \fbox{$\mathbb{B}[b]_s=ff$} è analogo al precedente.In questo caso
	per la regola \ifffSOS{} abbiamo:
	\[ \confSs{\ifABC{b}{S_1}{S_2}}{s}\Rar{}\confSs{{S_2}}{s}  \]e poichè
	sappiamo
	per \hi{} che \[ \confSs{S_2}{s}\Rar{*}s' \quad \textrm{qualunque sia s} \]
	si ha che $\confSs{\ifABC{b}{S_1}{S_2}}{s}\Rar{*}s''$
	\end{itemize}

	\end{itemize}
}
\newpage