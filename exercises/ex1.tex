\newcommand{\whilem}{\syntacticCategory{While\textsuperscript{-}}}
\newcommand{\exOne}{$\forall{}S\in\whilem,\forall{}s\in\states.\exists s'\in\states.\confSs{S}{s}\Rar{*}s'$}

\exercise{Esercizio 1}
{
	Let us consider the language \whilem{} defined by removing from \while{}
	the iterative command \texttt{while}.  Prove that for any 
	$S\in\whilem,$
	\begin{align*}
	\forall{}s\in\states.\exists s'\in\states.\confSs{S}{s}\Rar{*}s'.
	\end{align*}
}
{
	Per provare che \exOne{} utilizzo l'induzione sulla struttura di $S\in\stm$.
	\\I casi base sono:
	
	\begin{itemize}
	
	\item \casobase{S=\skipistr}: Se S=\skipistr{} allora per dimostrare lo
	statement è sufficiente applicare la regola dello \skipSOS{} e quindi preso
	un qualunque $s\in\states$ si ha:
	\begin{align*}
	\confSs{\skipistr}{s}\Rar{}s
	\end{align*}
	E quindi per determinismo si ha che $s=s'$ e quindi S termina in un passo.

	\item \casobase{S=\memupdate{x}{a}}: Se S=\memupdate{x}{a} con $x\in\var$
	e $a\in\num$ allora per dimostrare lo statement è sufficiente effettuare
	l'update della memoria e si ha che lo statement termina. Quindi preso un
	qualunque $s\in\states$ si ha:
	\begin{align*}
	\confSs{\memupdate{x}{a}}{s}\Rar{}s[x\rightarrow{}\mathbb{A}[a]_s]=s'
	\end{align*}
	E quindi S termina in un passo.
	
	\end{itemize}
	

	I casi induttivi invece sono la concatenazione e l'\texttt{if}.
	\begin{itemize}

	\item \casoinduttivo{S=\concat{$S_1$}{$S_2$}}: Se S=\concat{$S_1$}{$S_2$}
	con $S_1,S_2\in\whilem$, allora per ipotesi induttiva so che 
	$\forall{}s\in\states$, $\confSs{S_1}{s}\Rar{*}s'$ e 
	$\confSs{S_2}{s}\Rar{*}s''$, qualunque sia s. Poichè 
	$\confSs{S_1}{s}\Rar{*}s'$ (per ipotesi induttiva) allora 
	$\exists{}k\in\mathbb{N}$ tale che $\confSs{S_1}{s}\Rar{k}s'$ e quindi
	$\confSs{\concat{$S_1$}{$S_2$}}{s}\Rar{k}\confSs{S_2}{s'}$. Poichè
	dato un qualunque stato so che per ipotesi induttiva che $\confSs{S_2}{s'}$
	termina allora, in particolare, $\confSs{S_2}{s'}\Rar{*}s''$.

	\item \casoinduttivo{S=\ifABC{b}{S_1}{S_2}}: Se 
	S=\ifABC{b}{$S_1$}{$S_2$}, allora per ipotesi induttiva so che 
	$\forall{}s\in\states$, $\confSs{S_1}{s}\Rar{*}s'$ e 
	$\confSs{S_2}{s}\Rar{*}s''$. Quindi posso dimostrare che 
	\confSs{\ifABC{b}{S_1}{S_2}}{s}\Rar{*}$s'$ per casi:
	\begin{itemize}
	\item $\mathbb{B}[b]_s=tt$ allora 
	$\confSs{\ifABC{b}{S_1}{S_2}}{s}\Rar{}\confSs{{S_1}}{s}$ e poichè sappiamo
	per ipotesi induttiva che $\confSs{S_1}{s}\Rar{*}s'$ qualunque sia s si ha
	che $\confSs{\ifABC{b}{S_1}{S_2}}{s}\Rar{*}s'$

	\item $\mathbb{B}[b]_s=ff$ è analogo al precedente. In questo caso 
	$\confSs{\ifABC{b}{S_1}{S_2}}{s}\Rar{}\confSs{{S_2}}{s}$ e poichè sappiamo
	per ipotesi induttiva che $\confSs{S_2}{s}\Rar{*}s'$ qualunque sia s si ha
	che $\confSs{\ifABC{b}{S_1}{S_2}}{s}\Rar{*}s''$
	\end{itemize}

	\end{itemize}
}
