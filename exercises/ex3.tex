\newcommand{\repeatSbS}[3]{\texttt{repeat}~#1~\texttt{test}~#2~\texttt{then}~#3}

\newcommand{\PDef}{$\concat{$S_1$}{$\wbS{b}{(\concat{$S_2$}{$S_1$})}$}$}

\newcommand{\exThreeB}{$\repeatSbS{S_1}{b}{S_2}~\cong_{sos}~\PDef{}$}

\newcommand{\exThreeIff}{$
\confSs{\repeatSbS{S_1}{b}{S_2}}{s}\Rar{*}s'~\iff~
\confSs{\PDef{}}{s}\Rar{*}s'$}

\newcommand{\exThreeLtR}{$
\confSs{\repeatSbS{S_1}{b}{S_2}}{s}\Rar{*}s'~\Longrightarrow~
\confSs{\PDef{}}{s}\Rar{*}s'$}

\newcommand{\exThreeRtL}{$
\confSs{\PDef{}}{s}\Rar{*}s'~\Longrightarrow~
\confSs{\repeatSbS{S_1}{b}{S_2}}{s}\Rar{*}s'$}

\exercise{Esercizio 3}
{
	Assume that the language $\textbf{While}^{\textbf{+}}$ includes a new
	iterative command
	\begin{center}
	\repeatSbS{$S_1$}{$b$}{$S_2$}
	\end{center}
	where $b~\in~\bexp$ and $S_1,~S_2~\in~\stm$, whose informal semantics goes
	as follows:\\The following three steps are iteratively executed:
	\begin{enumerate}
	\item the command $S_1$ is executed
	\item the Boolean expression $b$ is evaluated; if $b$ evaluates to 
	\emph{false} then the loop is exited, otherwise the loop proceeds with step
	3
	\item command $S_2$ is executed
	\end{enumerate}
	\begin{itemize}
	\item [(a)] Define the small step operational semantics of this new
	iterative command.
	\item [(b)] Find a program \textbf{P} in the basic language \textbf{While} such that
	\begin{center}
	$\repeatSbS{$S_1$}{$b$}{$S_2$}~\cong_{sos}~P$
	\end{center}
	and formally prove this semantic equivalence.
	\end{itemize}
}
{
	\hspace*{\textwidth-}

	\textbf{Parte (a)}
	\begin{center}
	$\confSs{\repeatSbS{S_1}{b}{S_2}}{s}~\Rar{[repeat_{sos}]}~
	\confSs{\concat
	{$S_1$}
	{\ifABC{b}{\concat
	{$S_2$}
	{(\repeatSbS{$S_1$}{$b$}{$S_2$})}
	}{\skipistr}}
	}{s}$
	\end{center}
	L'idea di base è mimare esattamente quando descritto nella prima parte
	della consegna, eseguendo qualcosa di simile all'\emph{unfolding} della
	regola del while in semantica \texttt{SOS}. Per questo il comando
	\texttt{repeat} viene trasformato nella concatenazione del sottoprogramma
	$S_1$, che deve essere in ogni caso eseguito, con un \texttt{if}. Nel caso
	la valutazione della guardia dell'\texttt{if} (la stessa guardia
	specificata nel comando del \texttt{repeat}) sia falsa nello stato $s$
	allora il \texttt{repeat} evolve in \skipistr{}. Viceversa, se la guardia
	viene valutata a \texttt{true}, allora il \texttt{repeat} evolve nella
	concatenazione tra il sottoprogramma $S_2$ e la ripetizione del 
	\texttt{repeat}, in modo tale da permettere la ripetizione iterativa.\\

	\textbf{Parte (b)}
	\begin{center}
	$P~=~\PDef{}$
	\end{center}
	Per dimostrare che:
	\begin{center}
	\exThreeB{}
	\end{center}
	Dimostro che:
	\begin{center}
	\exThreeIff{}
	\end{center}
	Quindi posso dimostrare le due implicazioni del ``se e solo se''
	separatamente.

	\textbf{Implicazione $\Longleftarrow$}\\
	Devo dimostrare che:
	\begin{center}
	se \exThreeRtL{}
	\end{center}
	Dal momento che per ipotesi $\confSs{\PDef{}}{s}\Rar{*}s'$ allora so che
	$\exists{}~n~\in~\mathbb{N}~tale~che~\confSs{\PDef{}}{s}\Rar{*}s'$ quindi
	posso dimostrare sulla lunghezza $n$ della derivazione che tale
	implicazione vale.\\
	Poichè, sempre per ipotesi, suppongo che $\confSs{\PDef{}}{s}\Rar{n}s'$
	allora per il \textbf{lemma di composizione} $\exists{}~k_0~\in~\mathbb{N},~
	\confSs{S_1}{s}\Rar{k_0}s''~e~
	\confSs{\wbS{b}{(\concat{$S_2$}{$S_1$})}}{s''}\Rar{n-k_0}s'$.\\
	Quindi al posto di dimostrare:
	\begin{center}
	se \exThreeRtL{}
	\end{center}
	dimostro che:
	\begin{center}
	se $\confSs{\wbS{b}{(\concat{$S_2$}{$S_1$})}}{s''}\Rar{n-k_0}s'~
	\Longrightarrow~\confSs{\repeatSbS{S_1}{b}{S_2}}{s}\Rar{*}s'$
	\end{center}
	con $\confSs{S_1}{s}\Rar{k_0}s''$
	Per fare ciò sfrutto, come detto precedentemente, l'induzione su n.

	\casobase{n=3} In questo caso, se vogliamo che 
	$\confSs{\wbS{b}{(\concat{$S_2$}{$S_1$})}}{s''}$ arrivi ad una
	configurazione finale in 3 passi allora, sfruttando la regola del 
	\texttt{while} in semantica \texttt{SOS}, $\mathbb{B}[b]_{s''}$ deve essere
	necessariamente valutato a false. Quindi otteniamo:
	\begin{center}
	$\confSs{\wbS{b}{(\concat{$S_2$}{$S_1$})}}{s''}\Rar{}$\\
	$\confSs{\ifABC{b}{(\concat{\concat{$S_2$}{$S_1$}}
	{\wbS{b}{(\concat{$S_2$}{$S_1$})}})}{\skipistr}}{s''}\Rar{}$\\
	$\confSs{\skipistr}{s''}\Rar{}s''$
	\end{center}
	Nel lato destro dell'implicazione invece troviamo:
	\begin{center}
	$\confSs{\repeatSbS{S_1}{b}{S_2}}{s}\Rar{}$\\
	$\confSs{\concat{$S_1$}{\ifABC{b}{(\concat{$S_2$}
	{(\repeatSbS{$S_1$}{$b$}{$S_2$})})}{\skipistr}}}{s}$
	\end{center}
	Poichè per ipotesi sappiamo che $\confSs{S_1}{s}\Rar{k_0}s''$ allora
	per il \textbf{lemma di composizione} posso affermare che 
	\begin{center}
	$\confSs{\concat{$S_1$}{\ifABC{b}{(\concat{$S_2$}
	{(\repeatSbS{$S_1$}{$b$}{$S_2$})})}{\skipistr}}}{s}\Rar{k_0}$\\
	$\confSs{
		\ifABC{b}{(\concat{$S_2$}{(\repeatSbS{$S_1$}{$b$}{$S_2$})})} 
		{\skipistr}}{s''}$
	\end{center}
	Per ipotesi inoltre sappiamo che $\mathbb{B}[b]_{s''}=ff$ quindi:
	\begin{center}
	$\confSs{
		\ifABC{b}{(\concat{$S_2$}{(\repeatSbS{$S_1$}{$b$}{$S_2$})})} 
		{\skipistr}}{s''}\Rar{}$\\
	$\confSs{\skipistr}{s''}\Rar{}s''$
	\end{center}
	Ottenedo lo stesso stato finale quindi dell'implicazione di partenza.

	\casoinduttivo{n>3} In questo caso sappiamo necesariamente che 
	$\mathbb{B}[b]_{s''}=tt$ quindi nella parte destra dell'implicazione
	troviamo:


	\textbf{Implicazione $\Longrightarrow$}\\
	Devo dimostrare che:
	\begin{center}
	\exThreeLtR{}
	\end{center}
	Dal momento che $\confSs{\repeatSbS{S_1}{b}{S_2}}{s}\Rar{*}s'$ allora so che
	$\exists{}~n~\in~\mathbb{N}~tale~che~
	\confSs{\repeatSbS{S_1}{b}{S_2}}{s}\Rar{n}s'$ quindi posso dimostrare sulla
	lunghezza $n$ della derivazione che tale implicazione vale.

}