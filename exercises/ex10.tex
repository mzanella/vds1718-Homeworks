\newcommand{\dSwb}[2]
{
    \mbox{\normalfont{\texttt{do }}}
    #1
    \textrm{\normalfont{\texttt{ while }}}
    #2
}
\newcommand{\exTen}
{
	$ \wbS{b}{S} ~\cong{}_{ds}~$
	{$\ifABC{b}{(\dSwb{S}{b})}{\skipistr}$}
}
\newcommand{\Sds}{\dsCtxt{S}}
\newcommand{\mineq}{\sqsubseteq}
\newcommand{\unionsem}[1]{\bigsqcup\limits_{n\geq{}0}}

\exercise{Esercizio 10}
{
	Prove or disprove the following semantic equivalence:
	\begin{center}
		\exTen{}
	\end{center}
	where $b\in\bexp~and~S\in\stm$
}
{
    Per dimostrare che vale:
    \begin{center}
    \exTen{}
    \end{center}
    Devo dimostrare che:
    \begin{center}
    $\dsCtxt{\wbS{b}{S}}~=~
    \dsCtxt{\ifABC{b}{(\dSwb{S}{b})}{\skipistr}}$
    \end{center}
    Dove:
    \begin{center}
    $\dsCtxt{\wbS{b}{S}}=\fixp{F}$ con 
    $F~g~=~\cond{b}{g \circ{} \Sds{}}{\idDS}$ \\
    \end{center}
    e:
    \begin{center}
    $\dsCtxt{\ifABC{b}{(\dSwb{S}{b})}{\skipistr}}~=$\\$=~
    \cond{b}{\dsCtxt{\dSwb{S}{b}}}{\dsCtxt{\skipistr}}~=$\\$=~
    \cond{b}{\dsCtxt{\dSwb{S}{b}}}{\idDS}$
    \end{center}
    Devo prima di tutto, quindi, definire $\dsCtxt{\dSwb{S}{b}}$:
    \begin{center}
    $\dsCtxt{\dSwb{S}{b}}=\fixp{G}$ con 
    $G~g~=~\cond{b}{g}{\idDS} \circ \Sds$
    \end{center}
    Quindi devo dimostare che vale:
    \begin{center}
    $\fixp{F}~=~\cond{b}{\fixp{G}}{\idDS}$
    \end{center}
    Per fare ciò posso dimostrare separatamente le due seguenti affermazioni:
    \begin{enumerate}
    \item $\fixp{F}~\mineq~\cond{b}{\fixp{G}}{\idDS}$
    \item $\cond{b}{\fixp{G}}{\idDS}~\mineq~\fixp{F}$
    \end{enumerate}

    \textbf{Prima parte $\fixp{F}~\mineq~\cond{b}{\fixp{G}}{\idDS}$}\\
    Per dimostrare che vale $\fixp{F}~\mineq~\cond{b}{\fixp{G}}{\idDS}$
    utilizzo il \textbf{Fixed point induction lemma} che afferma
    che, date $f$ e $d$:
    \begin{center}
    se $fd~\mineq~d~\Longrightarrow~\fixp{f}~\mineq~d$
    \end{center}
    istanziato con:
    \begin{center}
    $f~=~F$ \\
    $d~=~\cond{b}{\fixp{G}}{\idDS}$
    \end{center}
    Quindi:
    \begin{center}
    se $F(\cond{b}{\fixp{G}}{\idDS})~\mineq~\cond{b}{\fixp{G}}{\idDS}~
    \Longrightarrow~\fixp{F}~\mineq~\cond{b}{\fixp{G}}{\idDS}$
    \end{center}
    Per dimostrare che $F(\cond{b}{\fixp{G}}{\idDS})~\mineq~
    \cond{b}{\fixp{G}}{\idDS}$ dimostro che:
    \begin{center}
    $\forall{} s\in\states$ se $(F(\cond{b}{\fixp{G}}{\idDS}))s=s'~
    \Longrightarrow~(\cond{b}{\fixp{G}}{\idDS})s=s'$
    \end{center}
    Procedo quindi per casi:\\
    \caso{\evalBbs{b}{s}=ff} allora in questo caso si ha:
    \begin{center}
    $(F(\cond{b}{\fixp{G}}{\idDS}))s~=$\\$=~
    (\cond{b}{(\cond{b}{\fixp{G}}{\idDS}) \circ{} \Sds{}}{\idDS})s~=~
    (\idDS)s~=s$
    \end{center}
    Dall'altra parte similarmente si ha:
    \begin{center}
    $(\cond{b}{\fixp{G}}{\idDS})s~=~(\idDS)s~=~s$
    \end{center}
    \caso{\evalBbs{b}{s}=tt} in questo caso, invece, si ha:
    \begin{center}
    $(F(\cond{b}{\fixp{G}}{\idDS}))s~=~(\cond{b}{(\cond{b}{\fixp{G}}{\idDS})
    \circ{} \Sds{}}{\idDS})s~=$\\$=~(\cond{b}{\fixp{G}}{\idDS} \circ{} \Sds{})s$
    \end{center}
    Dall'altra parte abbaimo quindi:
    \begin{center}
    $(\fixp{G})s~=~(G(\fixp{G}))s~=~(\cond{b}{\fixp{G}}{\idDS} \circ \Sds)s$
    \end{center}
    Quindi vale $\fixp{F}~\mineq~\cond{b}{\fixp{G}}{\idDS}$

    \textbf{Seconda parte $\cond{b}{\fixp{G}}{\idDS}~\mineq~\fixp{F}$}\\
    Per dimostrare che vale $\cond{b}{\fixp{G}}{\idDS}~\mineq~\fixp{F}$ ragiono
    per casi:\\
    \caso{\evalBbs{b}{s}=ff} in questo caso si ha:
    \begin{center}
    $(\cond{b}{\fixp{G}}{\idDS})s~=~(\idDS)s~=~s$
    \end{center}
    e dall'altra parte:
    \begin{center}
    $(\fixp{F})s~=~(F(\fixp{F}))s~=~(\cond{b}{\fixp{F} \circ{} \Sds{}}{\idDS})s
    ~=~(\idDS)s~=~s$
    \end{center}
    Arrivando quindi allo stesso stato finale.\\
    \caso{\evalBbs{b}{s}=tt} in questo caso ho che:
    \begin{center}
    $(\cond{b}{\fixp{G}}{\idDS})s~=~(\fixp{G})s$
    \end{center}
    e quindi devo dimostrare che $(\fixp{G})s~\mineq{}~(\fixp{F})s$ con
    $\evalBbs{b}{s}=tt$. Per fare questo utilizzo il \textbf{teorema di
    Knaster-Tarsky} e procedendo per induzione. Knaster-Tarsky afferma che:
    \begin{center}
    $\fixp{G}~=~\unionsem{n}G^{n}\bot$
    \end{center}
    Quindi dimostro per induzione su $n$ che vale:
    \begin{center}
    $\unionsem{n}G^{n}\bot~\mineq~\fixp{F}$
    \end{center}

    \casobase{n=0} In questo caso si ha che:
    \begin{center}
    $G^{n}\bot~=~G^{0}\bot~=~\bot~\mineq~\fixp{F}$
    \end{center}
    Per definizione di $\bot$.

    \casoinduttivo{n>0} Devo quindi dimostrare che:
    \begin{center}
    $G^{n}\bot~\mineq~\fixp{F}$ con $n~>~0$
    \end{center}
    Sapendo che per ipotesi induttiva vale:
    \begin{center}
    $\forall~t\in\mathbb{N},~t<n~G^{t}\bot~\mineq~\fixp{F}$
    \end{center}
    Sviluppo quindi $(G^{n}\bot)s$:
    \begin{center}
    $(G^{n}\bot)s~=~(G(G^{n-1}\bot))s~=~
    (\cond{b}{G^{n-1}\bot}{\idDS} \circ \Sds)s$
    \end{center}
    Dall'altra parte invece:
    \begin{center}
    $(\fixp{F})s~=~(F(\fixp{F}))s~=~(\cond{b}{\fixp{F} \circ{} \Sds{}}{\idDS})s~
    =~(\fixp{F} \circ{} \Sds{})s$
    \end{center}
    Ho quindi 2 sottocasi a seconda della valutazione di \Sds{} in $s$:
    \begin{itemize}
    \item \caso{\Sds{}s~=~undef} allora entrambi i costrutti evolveranno ad 
    $undef$ poichè $undef$ composto qualsiasi altro costrutto si valuterà ad 
    $undef$

    \item \caso{\Sds{}s~=~s'} allora si avrà:
    \begin{center}
    $(\cond{b}{G^{n-1}\bot}{\idDS} \circ \Sds)s~=~
    (\cond{b}{G^{n-1}\bot}{\idDS})s'$
    \end{center}
    e
    \begin{center}
    $(\fixp{F} \circ{} \Sds{})s~=~(\fixp{F})s'$
    \end{center}
    Abbiamo quindi altri 2 sottocasi a seconda della valutazione della guardia 
    $b$ in $s'$
        \begin{itemize}
        \item \caso{\evalBbs{b}{s}=ff} in questo caso abbiamo 
        \begin{center}
        $(\cond{b}{G^{n-1}\bot}{\idDS})s'~=~(\idDS)s'~=~s'$
        \end{center}
        e
        \begin{center}
        $(\fixp{F})s'~=~(F(\fixp{F}))s'~=~
        (\cond{b}{\fixp{F} \circ{} \Sds{}}{\idDS})s'~=~(\idDS)s'~=~s'$
        \end{center}
        Arrivando quindi allo stesso stato finale $s'$

        \item \caso{\evalBbs{b}{s}=tt} in questo caso abbiamo:
        \begin{center}
        $(\cond{b}{G^{n-1}\bot}{\idDS})s'~=~(G^{n-1}\bot)s'$
        \end{center}
        Poichè per ipotesi induttiva sappiamo che:
        \begin{center}
        $\forall~t\in\mathbb{N},~t<n~G^{t}\bot~\mineq~\fixp{F}$
        \end{center}
        e che $n-1<n$ allora $(G^{n-1}\bot)s'~\mineq~(\fixp{F})s'$
        \end{itemize}
    \end{itemize}
    E quindi vale $\cond{b}{\fixp{G}}{\idDS}~\mineq~\fixp{F}$
}
\newpage